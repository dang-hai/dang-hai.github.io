%%%%%%%%%%%%%%%%%%%%%%%%%%%%%%%%%%%%%%%%%
% Medium Length Professional CV
% LaTeX Template
% Version 2.0 (8/5/13)
%
% This template has been downloaded from:
% http://www.LaTeXTemplates.com
%
% Original author:
% Rishi Shah 
%
% Important note:
% This template requires the resume.cls file to be in the same directory as the
% .tex file. The resume.cls file provides the resume style used for structuring the
% document.
%
%%%%%%%%%%%%%%%%%%%%%%%%%%%%%%%%%%%%%%%%%

%----------------------------------------------------------------------------------------
%	PACKAGES AND OTHER DOCUMENT CONFIGURATIONS
%----------------------------------------------------------------------------------------

\documentclass{resume} % Use the custom resume.cls style
\usepackage{hyperref}

\usepackage[left=0.75in,top=0.6in,right=0.75in,bottom=0.6in]{geometry} % Document margins
\newcommand{\tab}[1]{\hspace{.2667\textwidth}\rlap{#1}}
\newcommand{\itab}[1]{\hspace{0em}\rlap{#1}}
\name{Hai Dang} % Your name
\address{Doctoral HCI Researcher} % Your address
\address{h.dang@uni-bayreuth.de} % Your phone number and email

\begin{document}

I'm doctoral researcher at the university of Bayreuth and part of the \href{https://www.hciai.uni-bayreuth.de/en/index.html}{HCI+AI research group}.
My research adopts an empirical user-centric perspective for new interactive systems for generative deep learning models.

%----------------------------------------------------------------------------------------
%	EDUCATION SECTION
%----------------------------------------------------------------------------------------

\begin{rSection}{Education}
{\bf University of Bayreuth, Bayreuth Germany } \hfill {\em September 2020 - July 2024} 
\\ Doctor of Philosophy
\\ Department of Computer Science \\
{\bf LMU Munich, Munich Germany } \hfill {\em October 2018 - July 2020} 
\\ Master of Science
\\ Department of Computer Science \hfill { GPA: 3.7 | (German Scale: 1.23) }
\\ Thesis: Representational Learning for Exploring\\Input Spaces in HCI\\
\\{\bf LMU Munich, Munich Germany} \hfill {\em October 2013 - July 2018} 
\\ Bachelor of Science
\\ Department of Computer Science \hfill { GPA: 3.7 | (German Scale: 1.38)}
\\ Thesis: Deep Conformance Checking \\ Efficient Estimation of Alignment Based Fitness.\\
\\{\bf Yonsei University, Seoul South Korea} \hfill {\em August 2016 - July 2017} 
\\ Bachelor of Science
\\ Department of Computer Science \hfill {\em Year Abroad} 

\end{rSection}

%----------------------------------------------------------------------------------------
%	TECHNICAL STRENGTHS SECTION
%----------------------------------------------------------------------------------------

\begin{rSection}{Technical Strengths}

    \begin{tabular}{ @{} >{\bfseries}l @{\hspace{6ex}} l }
    Python Libraries    \ & PyTorch (primarily), Pandas, Numpy, Tensorflow (occasionally)\\
    Programming Languages: \ &  Python (primarily), JavaScript (for interactive applications), Java\\
    Frameworks:            \ &  React, Svelte, d3js, MEAN-/MERN-Stack \\
    DevOps:     \ & Docker, Nginx, Git \\
    OS:  \ & Unix Systems (primarily), Windows (occasionally)\\
    \end{tabular}
    
\end{rSection}

%--------------------------------------------------------------------------------
%    Projects And Seminars
%-----------------------------------------------------------------------------------------------
\begin{rSection}{Projects}

\begin{rSubsection}{Representational Learning for Exploring Input Spaces in HCI}{Summer Semester 2020}{Master Thesis}{}
\item Developed an interactive tool to analyze gesture elicitation studies in HCI.
\item Used a variational autoencoder (Kingma et al. 2014) to learn a two-dimensional representation for every gesture pose in a given data set.
\item Visualized a gesture map that consists of a grid of gesture pose templates. A sequence of gesture poses was represented as a two-dimensional path on the map.
\item Introduced new statistical analysis concepts that were projected on the gesture map to exploratively analyze the gestural input.
\item Introduced the K-means clustering algorithm to separate distinct behavior from gesture elicitations. The cluster centroids were based on the DTW Barycenter Averaging algorithm (Petitjean et al. 2011)
\item Evaluated the thesis by conducting an expert interview and an in-depth analysis of a previously published gesture elicitation data set.
\end{rSubsection}

\vfill
\pagebreak

\begin{rSubsection}{Big Data Science}{Winter Semester 2019}{Seminar: Group Project}{}
\item Goal: Ensemble five graph embedding models for link prediction on knowledge graphs.
\item Reimplemented the knowledge graph embedding model ConvE (Dettmers et al. 2018).
\item Improved computation time by 30\% by using adequate vectorization.
\item Managed and monitored the machine learning models and the remote training clouds using MlFlow and DVC.
\item The ensemble model was comparable to the state-of-the-art model RotatE (Sun et al. 2019) but didn't provide significant improvements in link prediction performance.
\end{rSubsection}

\begin{rSubsection}{Sketching with Hardware}{Summer Semester 2019}{Workshop: Group Project}{}
\item Topic: Design an interactive game.
\item Build an interactive beer-pong platform named Pongly.
\item The game consisted of two wooden rotating platforms, each driven by an electric motor. The speed of the electric motor was controlled by a potentiometer, which players could manipulate during the game to change the rotation speed. Each platform could hold six drinking cups. A LED light and a magnetic sensor in each slot were used to track and visualize whether a cup was placed in the slot. The color of the LED lights indicated which game mode was currently active, e.g., rapid-fire, bonus points, locked cups, and so on.
\item Programmed the game logic into an Arduino board that was attached to the platforms.
\item Worked with the laser cutter, 3D printer, and other fabrication tools to produce the components for the game.
\end{rSubsection}

\begin{rSubsection}{Evaluation of Consumer Grade BCI Devices}{Summer Semester 2019}{Seminar: Group Project}{}
\item Topic: Experiments with a low-cost EEG-Based Brain-Computer Interface (BCI) from mBrainTrain.
\item Conducted multiple experiments with an EEG-Cap to measure the cognitive workload level of individual study participants using the N-Back task.
\item Applied basic signal processing techniques on the raw EEG recordings to extract the alpha and theta frequencies that characterize the cognitive workload.
\item Trained multiple classifiers (Support Vector Machines, Neural Nets, Random Forrest, K Nearest Neighbors) to differentiate between various workload levels.
\end{rSubsection}

\begin{rSubsection}{Development of an Interactive Sleep Monitoring Device}{Summer Semester 2019}{Seminar: Group Project}{}
\item Goal: Conceptualize and developed an interactive sleep monitoring device.
\item Device featured various sensors, including a light sensor, microphone, dust sensor, and a thermometer. Besides basic alarm clock functionalities, users could provide subjective sleep assessments via built-in buttons. The device had a built-in WiFi module that continuously transmitted the sensor data to a backend server.
\item Built the analytics backend to collect and analyze sleep data.
\item Designed the communication protocol between device and backend.
\end{rSubsection}

\begin{rSubsection}{Power Efficient High Performance Computing}{Winter Semester 2018}{Seminar: Group Project}{}
\item Goal: Develop a machine learning model to predict the power consumption of a data center in Munich.
\item Studied the center's warm-water cooling system.
\item Identified a set of hand-crafted features which were combined with automatically computed features from the raw sensor recordings from the water pumps.
\item Developed a recurrent neural network model for the prediction of energy consumption.
\item Achievement: Won the class competition for most accurate predictions by employing an autoregressive recurrent neural network.
\end{rSubsection}

\begin{rSubsection}{Intelligent User Interfaces}{Winter Semester 2018}{Seminar: Group Project}{}
\item Goal: Develop various intelligent system prototypes.
\item Developed an Amazon Alexa voice application to navigate our universities' extracurricular activities web page. It was aimed at first-year undergraduate students who wanted to explore the offerings of our university. 
\item Developed a basic text summarization and auto-completion tool based on the NLTK library.
\item Experimented with the IBM Watson cloud to build a system that identifies craft beer bottles and recommends a matching sausage type.
\end{rSubsection}

\end{rSection}

%\break

%----------------------------------------------------------------------------------------
%	WORK EXPERIENCE SECTION
%----------------------------------------------------------------------------------------

\begin{rSection}{Work Experience}

\begin{rSubsection}{SWM, Munich}{August 2019 - November 2019}{Machine Learning Developer}{Working Student}
\item Collected and integrated electricity data from various 
transmission system operators
\item Evaluated different machine learning models to predict feed-in-management operations
\item Build and deployed an end-to-end machine learning solution (Currently used in production)
\end{rSubsection}

\begin{rSubsection}{Celonis, Munich}{October 2018 - February 2019}{Software Developer}{Working Student}
\item Designed and implemented the python programming interface for the Celonis Business Intelligence Cloud Platform. (Currently used in production)
\item Developed a data pushing pipeline to programmatically analyze event data.
\end{rSubsection}

\begin{rSubsection}{Celonis, Munich}{October 2017 - October 2018}{Data Analyst}{Working Student}
\item Integrated custom data pools
\item Developed business analyses and gave data science workshops
\item Developed tools to automate and administrate the Celonis Process Mining Platform
\end{rSubsection}

\begin{rSubsection}{LMU Munich, Munich}{April 2016 - August 2016}{Tutor - Multi-Media-Programming}{Working Student}
\item Created and presented material for the lecture
\item Assisted the course participants during their programming assignments
\end{rSubsection}

\vfill
\pagebreak

\begin{rSubsection}{FeldM, Munich}{April 2015 - April 2016}{Java Software Developer}{Working Student}
\item Developed a tool to monitor and evaluate different social media channels
\item Extended an internal management tool to track working hours
\end{rSubsection}

\end{rSection}
%----------------------------------------------------------------------------------------
% Course Work
%----------------------------------------------------------------------------------------
\begin{rSection}{Selected Relevant Coursework} \itemsep -3pt
{\bf Data Processing and Analysis:}\\
Deep Learning Algorithms, Machine Learning, Knowledge Discovery in Databases, Probability and Measure Theory, Big Data Management\\

{\bf Human-Computer-Interaction:}\\
Advanced topics on HCI, Intelligent User Interfaces, Psycho-physiological Computing, Human-Computer-Interaction, Sketching with Hardware, Development of interactive media systems\\

{\bf General Software Development:}\\
Data Structures and Algorithms, Software Architecture, Software Testing, Agile Development

\end{rSection}

\end{document}
